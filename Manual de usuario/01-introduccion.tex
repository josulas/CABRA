\section{Introducción} \label{introduccion}

La audiometría es un estudio médico realizado con el objetico de evaluar la integridad funcional del sistema auditivo.
La audiometría tonal, en particuarl, es una prueba que se realiza para determinar la capacidad de una persona para escuchar sonidos a diferentes frecuencias y volúmenes.
El objetivo de este estudio es obtener un audiograma, el cual representa los límites audibles (expresados en [dbHL], decibelios normalizados por normoacusia) del paciente para distintos valores de frecuencia de sonido.
La audiometría tonal se realiza en ambos oídos, para determinar la capacidad auditiva de cada oído por separado.
El audiómetro es un dispositivo que genera tonos puros a diferentes frecuencias y volúmenes, reproducidos a través de auriculares \textit{over-ear} que se colocan en la cabeza del paciente y que permiten escuchar los tonos generados.
Mediante un barrido de amplitudes y frecuencias, se obtiene el audiograma del paciente analizando en qué condiciones se percibe y se deja de percibir el estímulo.
Generalmente, es el paciente quien debe indicar si escucha o no los tonos generados por el audiómetro.
Sin embargo, en pacientes que no pueden comunicarse verbalmente, como los bebés, se deben considerar otras técnicas para obtener el audiograma.

El potencial evocado auditivo del tronco cerebral (ABR, por sus siglas en inglés) es una técnica que permite evaluar la integridad de las vías auditivas desde el oído interno hasta el tronco cerebral.
El ABR es un potencial eléctrico generado por la actividad de las neuronas del nervio auditivo y del tronco cerebral en respuesta a un estímulo acústico.
Mediante el análisis de su morfología, detectando latencia y amplitud de ondas tipificadas, es posible detectar la percepción sonora.
Además, se puede evaluar la integridad de las vías auditivas y determinar la presencia de patologías en el sistema auditivo, mediante la obtención de potenciales evocados en respuesta a estímulos transmitidos por vía aérea o por vía ósea.
Esta es una técnica no invasiva, que no requiere la colaboración del paciente y que puede ser utilizada en pacientes de cualquier edad, incluyendo recién nacidos.

CABRA \textit{(Computarized Auditory Brainstem Response Audiometry)} es un sistema de audiometría automática que utiliza la técnica de ABR para obtener el audiograma del paciente.
El sistema CABRA se basa en la generación de estímulos sonoros a través de auriculares (de conducción aérea u ósea), y la medición de la respuesta del sistema auditivo a través de electrodos colocados en la cabeza del paciente.
El objetivo de este sistema es obtener el audiograma del paciente de manera automática, sin necesidad de la colaboración del paciente.
El siguiente manual describe el funcionamiento del sistema CABRA, detallando los pasos necesarios para realizar una audiometría automática y obtener el audiograma en forma semiautomatizada.