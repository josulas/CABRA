\section{Marco teórico} \label{marco_teorico}

\subsection{Potencial evocado auditivo}

El potencial evocado auditivo es una señal eléctrica generada por el tronco encefálico en respuesta a un estímulo auditivo.

\subsection{Adquisición de la señal fisiológica}

Para obtener una señal de potencial evocado auditivo mediante \textit{Auditory Brainstem Response} (ABR), se deben
colocar tres electrodos en el paciente de la siguiente manera:

\begin{itemize}
    \item Electrodo de activo ($V_{in}^{+}$): en la frente del paciente.
    \item Electrodo de tierra ($V_{in}^{-}$): en la apófisis mastoides ipsilateral del paciente.
    \item Electrodo de referencia (neutro): en la apófisis mastoides contralateral del paciente.
\end{itemize}

La señal de ABR se obtiene midiendo la diferencia de potencial entre los electrodos $V_{in}^{+}$ y $V_{in}^{-}$, y
tomando como referencia el electrodo neutro.
Esta señal es particularmente ruidosa, y su amplitud es del orden de los 500 [nV]. Por ello, se la debe filtrar y
amplificar analógicamente previo a su tratamiento digital \cite{shojaeemend_automated_2018}.
Se suele utlizar un filtro pasabandas de 100 a 3000 [Hz] para eliminar ruido de baja y alta frecuencia.
Para amplificar, como se explicará más adelante, se utilizará una combinación de amplificadores de instrumentación y
amplificadores operacionales para lograr una ganancia total de 6500.
