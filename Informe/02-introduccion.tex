\section{Introducción} \label{introduccion}

La audiometría es un estudio médico funcional que tiene como objetivo evaluar la percepción sonora de un paciente. Es de particular interés clínico, ya que permite detectar y caracterizar distintos tipos de hipoacusias.

Específicamente, se busca detectar a partir de qué intensidad de sonido (comúnmente llamada SPL, por las siglas en inglés de \textit{Sound Pressure Level}) en decibelios [dB] comienza a ser perceptible un sonido de una frecuencia específica en Hertz [Hz], a distintos niveles del procesamiento auditivo humano. Para ello, se estimula al paciente con señales auditivas en un barrido de frecuencias y SPL, pidiendo que se indique la intensidad para la que cada frecuencia deja de ser perceptible. De esta manera, se genera un gráfico denominado audiograma, que indica los límites perceptibles en dB vs. Hz.

El estímulo auditivo utilizado suele ser una secuencia de clics cortos, cuya forma de onda es senoidal pura y de frecuencia controlada. Este sonido se hace llegar al paciente por vía aérea y vía ósea, para poder así discernir entre hipoacusias conductivas (generadas por patologías en el oído medio) y neurológicas (por deficiencias en el oído interno, nervio auditivo, o el sistema nervioso central). Se utilizan audífonos para la conducción aérea, y dispositivos vibrantes presionados contra las sienes para la ósea.

Si bien la audiometría tradicional requiere de la participación activa del paciente, necesitando que se indique expresamente la percepción de cada estímulo, existen técnicas que permiten inferir esta condición a nivel de vías de conducción en distintas partes del encéfalo, de forma objetiva e independiente del sujeto y su capacidad de expresarse (o intenciones de colaborar en el ensayo clínico).

Una de las más utilizadas consiste en el procesamiento de señales de respuesta auditiva del tronco encefálico (ABR, por sus siglas en inglés para \textit{Auditory Brainstem Response}). Esta se obtiene midiendo la diferencia de potencial entre dos electrodos contralaterales, colocados en las apófisis mastoides, y tomando como referencia la frente.
Debido a que se trata de una señal particularmente ruidosa, se la procesa promediando los resultados de múltiples mediciones para iguales condiciones experimentales (esto es, misma frecuencia y amplitud de clic) \cite{young-ABR}.

Como resultado, se obtiene una forma de onda tipificada, con picos y valles conocidos \cite{shojaeemend_automated_2018}. Mediante técnicas de procesamiento digital, se puede estimar la percepción sonora para cada frecuencia en función del SPL, independizándose de la participación del paciente. \cite{silva_objective_2009}.

Por otro lado, según un reciente artículo de \textit{Market Research and News} \cite{news_abr_nodate}, el mercado de dispositivos de audiometría basada en ABR se encuentra a la alza. Estos equipos son de particular interés para los casos en los que el paciente no puede acceder a una audiometría convencional que requiera de su participación, como es el caso de los infantes y neonatos.

Es especialmente relevante detectar hipoacuasias en infantes de manera temprana, debido a que de ser necesario el uso de implantes cocleares, estos se adoptan en forma óptima durante los primeros 12 a 36 meses.
Por este motivo, muchos países están implementando programas de evaluación auditiva generalizada, contribuyendo así al interés del mercado en productos aptos para infantes.

A día de hoy, los principales desarrolladores de este tipo de dispositivos son Welch Allyn y Natus, ambas basadas en los EE. UU. Sin embargo, se prevé un crecimiento de la industria para los próximos 10 años en el mercado sudamericano, particularmente en México, Brasil y Argentina \cite{news_abr_nodate}. Para este mercado, es vital considerar la accesibilidad en costo, para permitir así su uso en zonas carenciadas.
Mientras que en Norteamérica esto no es una prioridad, en Argentina y la región es uno de los puntos clave.

En este contexto, surge CABRA (Computer Automated Brainstem Response Audiometry), un sistema de audiometría basado en ABR que busca ser accesible en costo y fácil de usar. Mediante este dispositivo, se pueden realizar un estudio completo de audiometría registrando las respuestas del tronco encefálico a estímulos auditivos, sugerir los umbrales de percepción y generar un audiograma. Además, se busca que el sistema sea modular, permitiendo la incorporación de mejoras y adaptaciones en el futuro. Esto permitirá incorporar funcionalidades extra, ergonomía superior y responder a las necesidades de los usuarios.

\subsection{Objetivos del proyecto} \label{objetivos}

\textbf{Objetivos de Mímina}

\begin{itemize}
    \item Proveer al usuario de una interfaz gráfica para el control de los estímulos y la visualización de los resultados
    \item Permitir generación de estímulos por vía aérea y ósea, considerando la elección realizada
    \item Contar con una correcta adquisición de la señal ABR mediante un preprocesamiento analógico y un posterior filtrado digital
    \item Integrar todo el hardware de adquisición en una carcasa ergonómica y compacta
\end{itemize}

\newpage

\textbf{Objetivos de Máxima}

\begin{itemize}
    \item Proveer al usuario de un posprocesamiento adicional de la señal de ABR para la detección de los umbrales de percepción
    \item Lograr la automatización de la generación de estímulos necesarios para completar una audiometría
    \item Minimizar del tiempo de estudio mediante selección inteligente de estímulos
    \item Integrar toda la adquisición y generación de estímulos en el dispositivo, independizando al paciente del profesional clínico
\end{itemize}
