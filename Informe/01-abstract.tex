\section{Abstract} \label{abstract}

Este trabajo describe el diseño e implementación de un sistema automatizado de audiometría computarizada basado en
la detección y análisis de potenciales evocados auditivos.

El estudio de audiometría busca detectar los umbrales de amplitud sonora a partir de los cuales un sonido es
perceptible para un paciente ante diferentes frecuencias de sonido.
Para ello, se generan estímulos auditivos correspondientes a frecuencias específicas, a diferentes amplitudes, hasta
encontrar el punto umbral.
Técnicas más avanzadas implican el análisis de potenciales evocados auditivos para automatizar la detección de
dichos límites.
Se investigó acerca de los productos más utilizados para realizar este tipo de estudios, y se encontró que los
dispositivos comerciales disponibles son de un elevado costo, además de que no existen alternativas de producción
nacional.

Nuestro proyecto, CABRA (Computer Automated Brainstem Response Audiometry), tiene como objetivo realizar audiometrías
basadas en análisis de potenciales evocados auditivos del tronco encefálico a un bajo costo.
Para ello, se diseñó un sistema de hardware basado en un microcontrolador ESP32-S2, el cual recibe una señal
proveniente del tronco encefálico del paciente (mediante electrodos colocados superficialmente), previamente
filtrada y amplificada, y la digitaliza para su posterior procesamiento.
Luego, las señales recibidas son procesadas digitalmente por un software de escritorio, el cual se encarga de
realizar el estudio completo de audiometría (generando los estímulos auditivos correspondientes) analizando las
señales fisiológicas obtenidas.

Este sistema es modular, amigable al usuario, rápido, y ampliamente configurable.
Además de proveer una herramienta para evaluar la integridad de las vías auditivas del paciente, CABRA permite
acelerar el estudio de audiometría al guiar al usuario en la búsqueda de los umbrales de percepción mediante un
sistema experto.

A futuro, como posibles mejoras, podría modificarse el sistema de adquisición de señales, junto con el de generación
de estímulos, para disminuir la latencia en el muestreo.
Con respecto al software, el sistema experto podría optimizarse mediante algoritmos de machine learning.
Además, la presentación de la carcasa podría ser más compacta y ergonómica.

