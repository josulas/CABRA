\section{Discusion} \label{discusion}
Si bien la CABRA cumple con las expectativas planteadas para el proyecto de Instrumentación Biomédica II, está lejos de ser un producto terminado. Existen diversas mejoras que podrían implementarse para optimizar la obtención de los ABR, procurando mantener el costo y el tamaño del proyecto lo más bajos posible.

Entre estas mejoras, se sugiere reemplazar la ESP32 por un microprocesador con especificaciones superiores. Entre ellas, una mayor velocidad de comunicación, un ADC mejor comportado, y que incluya o permita la incorporación de una placa de audio para lograr una mayor sincronización entre la emisión de los pulsos y la captura de la señal. En cuanto al ADC, otra opción viable es la de incorporar uno externo, que se comunique con el microcontrolador mediante el protocolo I2C \textit{Inter-Integrated Circuit Protocol} o similar. Tal sugerencia representa uno de los aspectos más importantes, ya que los conversores internos de microcontroladores como la ESP32 están muy por debajo de la calidad requerida para aplicaciones médicas \cite{espressif-systems-ESP32}.

En cuanto a la practicidad del dispositivo, es recomendable implementar comunicación inalámbrica (por ejemplo, a través de Bluetooth) para la transmisión de información. Junto con la reproducción de audio desde el microcontrolador, ambas mejoras permitirían que el paciente no tenga que ubicarse junto con el usuario especialista, sino que pueda moverse libremente e incluso estar dentro de un ambiente insonoro separado.

Finalmente, es importante recalcar que, si bien el algoritmo de clasificación realiza sugerencias adecuadas cuando la calidad de la señal es aceptable, es de gran relevancia reemplazarlo por un sistema más robusto. Para ello, se plantea emplear técnicas \textit{Deep Learning}, que permitan evaluar los resultados de la prueba con mayor granularidad e incluso detectar problemas en el dispositivo o la conexión del paciente. No obstante, tal tarea requiere de la recolección y marcación de un gran número de muestras, exigiendo un esfuerzo de magnitud considerable.

